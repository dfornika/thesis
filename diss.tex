%%%%%%%%%%%%%%%%%%%%%%%%%%%%%%%%%%%%%%%%%%%%%%%%%%%%%%%%%%%%%%%%%%%%%%
% Template for a UBC-compliant dissertation
% At the minimum, you will need to change the information found
% after the "Document meta-data"
%
%!TEX TS-program = pdflatex
%!TEX encoding = UTF-8 Unicode
 
%% The ubcdiss class provides several options:
%%   fogscopy
%%       set parameters to exactly how FoGS specifies
%%         * single-sided
%%         * page-numbering starts from title page
%%         * the lists of figures and tables have each entry prefixed
%%           with 'Figure' or 'Table'
%%       This can be tested by `\iffogscopy ... \else ... \fi'
%%   10pt, 11pt, 12pt
%%       set default font size
%%   oneside, twoside
%%       whether to format for single-sided or double-sided printing
%%   balanced
%%       when double-sided, ensure page content is centred
%%       rather than slightly offset (the default)
%%   singlespacing, onehalfspacing, doublespacing
%%       set default inter-line text spacing; the ubcdiss class
%%       provides \textspacing to revert to this configured spacing
%%   draft
%%       disable more intensive processing, such as including
%%       graphics, etc.
%%
 
% For submission to FoGS
\documentclass[
 fogscopy,
 onehalfspacing,
 11pt
 % ,letterpaperm
 ]{ubcdiss}
 
% For your own copies (looks nicer)
% \documentclass[balanced,twoside,11pt]{ubcdiss}
 
%%%%%%%%%%%%%%%%%%%%%%%%%%%%%%%%%%%%%%%%%%%%%%%%%%%%%%%%%%%%%%%%%%%%%%
%%%%%%%%%%%%%%%%%%%%%%%%%%%%%%%%%%%%%%%%%%%%%%%%%%%%%%%%%%%%%%%%%%%%%%
%%
%% FONTS:
%%
%% The defaults below configures Times Roman for the serif font,
%% Helvetica for the sans serif font, and Courier for the
%% typewriter-style font.  Configuring fonts can be time
%% consuming; we recommend skipping to END FONTS!
%%
%% If you're feeling brave, have lots of time, and wish to use one
%% your platform's native fonts, see the commented out bits below for
%% XeTeX/XeLaTeX.  This is not for the faint at heart.
%% (And shouldn't you be writing? :-)
%%
 
%% NFSS font specification (New Font Selection Scheme)
\usepackage{times,mathptmx,courier}
\usepackage[scaled=.92]{helvet}
 
%% Math or theory people may want to include the handy AMS macros
%\usepackage{amssymb}
%\usepackage{amsmath}
%\usepackage{amsfonts}
 
%% The pifont package provides access to the elements in the dingbat font.   
%% Use \ding{##} for a particular dingbat (see p7 of psnfss2e.pdf)
%%   Useful:
%%     51,52 different forms of a checkmark
%%     54,55,56 different forms of a cross (saltyre)
%%     172-181 are 1-10 in open circle (serif)
%%     182-191 are 1-10 black circle (serif)
%%     192-201 are 1-10 in open circle (sans serif)
%%     202-211 are 1-10 in black circle (sans serif)
%% \begin{dinglist}{##}\item... or dingautolist (which auto-increments)
%% to create a bullet list with the provided character.
\usepackage{pifont}
 
%%%%%%%%%%%%%%%%%%%%%%%%%%%%%%%%%%%%%%%%%%%%%%%%%%%%%%%%%%%%%%%%%%%%%%
%% Configure fonts for XeTeX / XeLaTeX using the fontspec package.
%% Be sure to check out the fontspec documentation.
%\usepackage{fontspec,xltxtra,xunicode} % required
%\defaultfontfeatures{Mapping=tex-text} % recommended
%% Minion Pro and Myriad Pro are shipped with some versions of
%% Adobe Reader.  Adobe representatives have commented that these
%% fonts can be used outside of Adobe Reader.
%\setromanfont[Numbers=OldStyle]{Minion Pro}
%\setsansfont[Numbers=OldStyle,Scale=MatchLowercase]{Myriad Pro}
%\setmonofont[Scale=MatchLowercase]{Andale Mono}
 
%% Other alternatives:
%\setromanfont[Mapping=tex-text]{Adobe Caslon}
%\setsansfont[Scale=MatchLowercase]{Gill Sans}
%\setsansfont[Scale=MatchLowercase,Mapping=tex-text]{Futura}
%\setmonofont[Scale=MatchLowercase]{Andale Mono}
%\newfontfamily{\SYM}[Scale=0.9]{Zapf Dingbats}
%% END FONTS
%%%%%%%%%%%%%%%%%%%%%%%%%%%%%%%%%%%%%%%%%%%%%%%%%%%%%%%%%%%%%%%%%%%%%%
%%%%%%%%%%%%%%%%%%%%%%%%%%%%%%%%%%%%%%%%%%%%%%%%%%%%%%%%%%%%%%%%%%%%%%
 
 
 
%%%%%%%%%%%%%%%%%%%%%%%%%%%%%%%%%%%%%%%%%%%%%%%%%%%%%%%%%%%%%%%%%%%%%%
%%%%%%%%%%%%%%%%%%%%%%%%%%%%%%%%%%%%%%%%%%%%%%%%%%%%%%%%%%%%%%%%%%%%%%
%%
%% Recommended packages
%%
\usepackage{checkend}   % better error messages on left-open environments
\usepackage{graphicx}   % for incorporating external images
 
%% booktabs: provides some special commands for typesetting tables as used
%% in excellent journals.  Ignore the examples in the Lamport book!
\usepackage{booktabs}
 
%% listings: useful support for including source code listings, with
%% optional special keyword formatting.  The \lstset{} causes
%% the text to be typeset in a smaller sans serif font, with
%% proportional spacing.
\usepackage{listings}
\lstset{basicstyle=\sffamily\tiny,showstringspaces=false,fontadjust}
 
%% The acronym package provides support for defining acronyms, providing
%% their expansion when first used, and building glossaries.  See the
%% example in glossary.tex and the example usage throughout the example
%% document.
%% NOTE: to use \MakeTextLowercase in the \acsfont command below,
%%   we *must* use the `nohyperlinks' option -- it causes errors with
%%   hyperref otherwise.  See Section 5.2 in the ``LaTeX 2e for Class
%%   and Package Writers Guide'' (clsguide.pdf) for details.
\usepackage[printonlyused,nohyperlinks]{acronym}
%% The ubcdiss.cls loads the `textcase' package which provides commands
%% for upper-casing and lower-casing text.  The following causes
%% the acronym package to typeset acronyms in small-caps
%% as recommended by Bringhurst.
 
% Adding Package glossary, in order to suppliment the acronym package used by default in the glossary section - APF
%\usepackage{glossary}
%\makeglossary
 
% APF - suggested update to glossaries, instead of glossary package.
\usepackage[xindy,toc]{glossaries}
\makeglossaries
 
% DJF - mhchem is used to format chemical symbols
\usepackage[version=3]{mhchem}

% DJF - paralist is used for in-paragraph enumerated lists (\inparaenum)
\usepackage{paralist}

% DJF - listing used for basic source code, minted for fancy syntax hilighting.
% \usepackage{listings}
% \usepackage{minted}

% DJF - longtable used for multi-page tables
\usepackage{longtable}

% DJF - lscape for landscape pages
\usepackage{pdflscape}

% DJF - colortbl for shading table cells
\usepackage{colortbl}

%%%%%%%%%%%%%%%%%%%%%%%%%%%%%%%%%%%%%%%%%%%%%%%%%%%%%%%%%%%%%%%%%%%%
%APF - Making a subscript command:
\usepackage{relsize}
\newcommand{\subscript}[1]{\raisebox{-0.25em}{\smaller #1}}
 
%% modified by APF, because it was annoying me. Original first.
%\renewcommand{\acsfont}[1]{{\scshape \MakeTextLowercase{#1}}}
\renewcommand{\acsfont}[1]{{\footnotesize #1}}

%% DJF - Use symbols for footnotes
\usepackage{footmisc}
\renewcommand{\thefootnote}{\fnsymbol{footnote}}

%% Recommended package for changing widths of margins, but do not use
%% unless absolutely necessary! - APF
%% \usepackage[margin=1.25in,top=1.25in,bottom=1.25in]{geometry}
 
%% color: add support for expressing colour models.  Grey can be used
%% to great effect to emphasize other parts of a graphic or text.
%% For an excellent set of examples, see Tufte's "Visual Display of
%% Quantitative Information" or "Envisioning Information".
\usepackage{color}
\definecolor{greytext}{gray}{0.5}
 
%% comment: provides a new {comment} environment: all text inside the
%% environment is ignored.
%%   \begin{comment} ignored text ... \end{comment}
\usepackage{comment}
 
%% The natbib package provides more sophisticated citing commands
%% such as \citeauthor{} to provide the author names of a work,
%% \citet{} to produce an author-and-reference citation,
%% \citep{} to produce a parenthetical citation.
%% We use \citeeg{} to provide examples
%\usepackage[numbers,authoryear,sort&compress,round,colon,super]{natbib}
%\usepackage[numbers,sort&compress,colon,super]{natbib}
%\newcommand{\citeeg}[1]{\citep[e.g.,][]{#1}}
 
%%%%%%%%%%%%%%%%%%%%%%%%%%%%%%%%%%%%%%
%APF - adding biblatex to replace natbib for code display purposes
\usepackage[hyperref=true,backref=true,natbib=true,sorting=none,backend=biber]{biblatex}
\bibliography{biblio.bib}
 
%% The titlesec package provides commands to vary how chapter and
%% section titles are typeset.  The following uses more compact
%% spacings above and below the title.  The titleformat that follow
%% ensure chapter/section titles are set in singlespace.
\usepackage[compact]{titlesec}
\titleformat*{\section}{\singlespacing\raggedright\bfseries\Large}
\titleformat*{\subsection}{\singlespacing\raggedright\bfseries\large}
\titleformat*{\subsubsection}{\singlespacing\raggedright\bfseries}
\titleformat*{\paragraph}{\singlespacing\raggedright\itshape}
 
%% The caption package provides support for varying how table and
%% figure captions are typeset.
\usepackage[format=hang,indention=-1cm,labelfont={bf},margin=1em]{caption}
 
%% url: for typesetting URLs and smart(er) hyphenation.
%% \url{http://...}
\usepackage{url}
\urlstyle{sf}   % typeset urls in sans-serif
 
 
%%%%%%%%%%%%%%%%%%%%%%%%%%%%%%%%%%%%%%
%APF - adding use listings for code display purposes
\usepackage{listings}
\lstset{language=Java}
%APF - adding packages for multi-row /multi-column spanning boxes in
%tables
\usepackage{multirow}
\usepackage{multicol}
 
%APF - adding packages float
\usepackage{float}
 
%APF - adding package textcomp to allow use of < and > characters in text.
\usepackage{textcomp}
 
%APF - adding package microtype to allow lines to be expanded to prevent character overruns.
\usepackage{microtype}
 
%%%%%%%%%%%%%%%%%%%%%%%%%%%%%%%%%%%%%%%%%%%%%%%%%%%%%%%%%%%%%%%%%%%%%%
%%%%%%%%%%%%%%%%%%%%%%%%%%%%%%%%%%%%%%%%%%%%%%%%%%%%%%%%%%%%%%%%%%%%%%
%%
%% Possibly useful packages: you may need to explicitly install
%% these from CTAN if they aren't part of your distribution;
%% teTeX seems to ship with a smaller base than MikTeX and MacTeX.
%%
%\usepackage{pdfpages}  % insert pages from other PDF files
%\usepackage{longtable} % provide tables spanning multiple pages
%\usepackage{chngpage}  % support changing the page widths on demand
%\usepackage{tabularx}  % an enhanced tabular environment
 
%% enumitem: support pausing and resuming enumerate environments.
%\usepackage{enumitem}
 
%% rotating: provides two environments, sidewaystable and sidewaysfigure,
%% for typesetting tables and figures in landscape mode.  
\usepackage{rotating}
 
%% subfig: provides for including subfigures within a figure,
%% and includes being able to separately reference the subfigures.
\usepackage{subfig}
 
%% ragged2e: provides several new new commands \Centering, \RaggedLeft,
%% \RaggedRight and \justifying and new environments Center, FlushLeft,
%% FlushRight and justify, which set ragged text and are easily
%% configurable to allow hyphenation.
%\usepackage{ragged2e}
 
%% The ulem package provides a \sout{} for striking out text and
%% \xout for crossing out text.  The normalem and normalbf are
%% necessary as the package messes with the emphasis and bold fonts
%% otherwise.
%\usepackage[normalem,normalbf]{ulem}    % for \sout
 
%%%%%%%%%%%%%%%%%%%%%%%%%%%%%%%%%%%%%%
%APF - trying out bibentry
%\usepackage{bibentry}
%\nobibliography{biblio}
 
 
%%%%%%%%%%%%%%%%%%%%%%%%%%%%%%%%%%%%%%%%%%%%%%%%%%%%%%%%%%%%%%%%%%%%%%
%% HYPERREF:
%% The hyperref package provides for embedding hyperlinks into your
%% document.  By default the table of contents, references, citations,
%% and footnotes are hyperlinked.
%%
%% Hyperref provides a very handy command for doing cross-references:
%% \autoref{}.  This is similar to \ref{} and \pageref{} except that
%% it automagically puts in the *type* of reference.  For example,
%% referencing a figure's label will put the text `Figure 3.4'.
%% And the text will be hyperlinked to the appropriate place in the
%% document.
%%
%% Generally hyperref should appear after most other packages
 
%% The following puts hyperlinks in very faint grey boxes.
%% The `pagebackref' causes the references in the bibliography to have
%% back-references to the citing page; `backref' puts the citing section
%% number.  See further below for other examples of using hyperref.
%% 2009/12/09: now use `linktocpage' (Jacek Kisynski): FoGS now prefers
%%   that the ToC, LoF, LoT place the hyperlink on the page number,
%%   rather than the entry text.
\usepackage[
   %letterpaper, % compile issues on this command
 bookmarks,bookmarksnumbered,%
   citebordercolor={0.8 0.8 0.8},filebordercolor={0.8 0.8 0.8},%
   linkbordercolor={0.8 0.8 0.8},%
   urlbordercolor={0.8 0.8 0.8},%
   %pagebackref,linktocpage,%
   plainpages=false, pdfpagelabels %Added by APFejes to fix page # issues.
   ]{hyperref}
%% The following change how the the back-references text is typeset in a
%% bibliography when `backref' or `pagebackref' are used
%\renewcommand\backrefpagesname{\(\rightarrow\) pages}
%\renewcommand\backref{\textcolor{greytext} \backrefpagesname\ }
 
%% The following uses most defaults, which causes hyperlinks to be
%% surrounded by colourful boxes; the colours are only visible in
%% PDFs and don't show up when printed:
%\usepackage[bookmarks,bookmarksnumbered]{hyperref}
 
%% The following disables the colourful boxes around hyperlinks.
%\usepackage[bookmarks,bookmarksnumbered,pdfborder={0 0 0}]{hyperref}
 
%% The following disables all hyperlinking, but still enabled use of
%% \autoref{}
%\usepackage[draft]{hyperref}
 
%% The following commands causes chapter and section references to
%% uppercase the part name.
%\renewcommand{\chapterautorefname}{Chapter}
%\renewcommand{\sectionautorefname}{Section}
%\renewcommand{\subsectionautorefname}{Section}
%\renewcommand{\subsubsectionautorefname}{Section}
 
 
%%%%%%%%%%%%%%%%%%%%%%%%%%%%%%%%%%%%%%%%%%%%%%%%%%%%%%%%%%%%%%%%%%%%%%
%%%%%%%%%%%%%%%%%%%%%%%%%%%%%%%%%%%%%%%%%%%%%%%%%%%%%%%%%%%%%%%%%%%%%%
%%
%% Some special settings that controls how text is typeset
%%
% \raggedbottom         % pages don't have to line up nicely on the last line
% \sloppy               % be a bit more relaxed in inter-word spacing
% \clubpenalty=10000    % try harder to avoid orphans
% \widowpenalty=10000   % try harder to avoid widows
% \tolerance=1000
 
%% And include some of our own useful macros
\input{macros}
 
 
%%%%%%%%%%%%%%%%%%%%%%%%%%%%%%%%%%%%%%%%%%%%%%%%%%%%%%%%%%%%%%%%%%%%%%
%%%%%%%%%%%%%%%%%%%%%%%%%%%%%%%%%%%%%%%%%%%%%%%%%%%%%%%%%%%%%%%%%%%%%%
%%
%% Document meta-data: be sure to also change the \hypersetup information
%%
 
\title{Mitochondrial Genome Variation in Healthy Aging}
%\subtitle{}
 
\author{Daniel John Fornika}
\previousdegree{Bachelor of Science, Simon Fraser University, 2006}
 
% What is this dissertation for?
\degreetitle{Master of Science}
 
\institution{The University Of British Columbia}
\campus{Vancouver}
 
\faculty{The Faculty of Graduate Studies}
\department{Medical Genetics}
\submissionmonth{August}
\submissionyear{2012}
 
%% hyperref package provides support for embedding meta-data in .PDF
%% files
\hypersetup{
 pdftitle={Daniel John Fornika - Thesis  (\today)},
 pdfauthor={Daniel John Fornika},
 pdfkeywords={Mitochondria, Aging}
}
 
%%%%%%%%%%%%%%%%%%%%%%%%%%%%%%%%%%%%%%%%%%%%%%%%%%%%%%%%%%%%%%%%%%%%%%
%%%%%%%%%%%%%%%%%%%%%%%%%%%%%%%%%%%%%%%%%%%%%%%%%%%%%%%%%%%%%%%%%%%%%%
%%
%% The document content
%%
 
%% LaTeX's \includeonly commands causes any uses of \include{} to only
%% include files that are in the list.  This is helpful to produce
%% subsets of your thesis (e.g., for committee members who want to see
%% the dissertation chapter by chapter).  It also saves time by
%% avoiding reprocessing the entire file.
%\includeonly{intro,conclusions}
%\includeonly{discussion}
 
\begin{document}

%%%%%%%%%%%%%%%%%%%%%%%%%%%%%%%%%%%%%%%%%%%%%%%%%%
%% From Thesis Components: Tradtional Thesis
%% <http://www.grad.ubc.ca/current-students/dissertation-thesis-preparation/order-components>
 
% Preliminary Pages (numbered in lower case Roman numerals)
%    1. Title page (mandatory)
\maketitle
 
%    2. Abstract (mandatory - maximum 350 words)
%% The following is a directive for TeXShop to indicate the main file
%%!TEX root = diss.tex

\chapter{Abstract}

Mitochondria are thought to play a role in the aging process through their production of reactive oxygen species (ROS), and their regulation of cell fate via senescence and apoptosis.  We hypothesize that genetic variation in the mitochondrial genome may explain a portion of the phenotypic variance in the development of long-term good health.  To test this hypothesis, we have performed genetic association tests on a set of common mitochondrial polymorphisms, in a study of 419 exceptionally healthy seniors (cases) and 415 population-based mid-life individuals (controls).

Variant discovery was performed using Sanger sequencing of 834 individuals for the 1.1 kb non-coding mitochondrial control region, and identified 277 SNPs present in at least one individual.  A set of 92 mitochondrial coding-region SNPs were chosen via pooled high-throughput sequencing, combined with a previously-published set of European-specific mitochondrial tag SNPs.

After filtering for minor-allele frequency of $> 10\%$, a set of nine control-region SNPs and seven coding-region SNPs were tested for association with healthy aging.  None showed a statistically-significant association signal.  Additionally, one control-region variant that had shown association in an Italian centenarian population was tested in our sample set, but the association was not replicated.


\cleardoublepage
 
\acresetall     % reset all acronyms used so far
 
%    3. Preface
\chapter{Preface}

Funding for these experiments was obtained from \ac{CIHR} by Angela Brooks-Wilson, in collaboration with the Marco Marra, Joseph Connors, Stephen Jones, Nhu Le and Graydon Meneilly.  The experiments described in this thesis are a part of the Genomics, Genetics and Gerontology (G\textsuperscript{3}) Study of Healthy Aging.

All experiments were concieved of and designed by Angela Brooks-Wilson.  This thesis is broadly composed of three sub-experiments.  They are: pooled next-generation sequencing of mitochondrial DNA, Sanger sequencing of individual mitochondrial control-regions, and determination of mitochondrial genotypes by Sequenom genotyping.

Dan Fornika performed the \ac{PCR} reactions necessary to provide template mitochondrial DNA for both pooled mitochondrial DNA sequencing and control-region Sanger sequencing.  Mitochondrial DNA pools were constructed by Dan Fornika.

Sanger sequencing was prepared by Julius Halaschek-Wiener with assistance from Dan Fornika.  Sanger sequencing was performed by the sequencing group of the \ac{BCGSC}.

Library construction and Illumina \ac{GA} sequencing was performed by the sequencing group of the \ac{BCGSC}.

Sequenom genotyping was performed by the McGill University/Genome Qu\'{e}bec Innovation Centre.

All bioinformatics and statistical analyses were performed by Dan Fornika, except for quality control of Sequenom genotype data, which was performed by Denise Daley.

\cleardoublepage
 
%    4. Table of contents (mandatory - list all items in the preliminary pages
%    starting with the abstract, followed by chapter headings and
%    subheadings, bibliographies and appendices)
\tableofcontents
\cleardoublepage        % required by tocloft package
 
%    5. List of tables (mandatory if thesis has tables)
\listoftables
\cleardoublepage        % required by tocloft package
 
%    6. List of figures (mandatory if thesis has figures)
\listoffigures
\cleardoublepage        % required by tocloft package
 
%    7. List of illustrations (mandatory if thesis has illustrations)
%    8. Lists of symbols, abbreviations or other (optional)
 
%    9. Glossary (optional)
%% The following is a directive for TeXShop to indicate the main file
%%!TEX root = diss.tex

\chapter{Glossary}

% use \acrodef to define an acronym, but no listing
\acrodef{UI}{user interface}
\acrodef{UBC}{University of British Columbia}

% The acronym environment will typeset only those acronyms that were
% *actually used* in the course of the document
\begin{acronym}
\acro{ANOVA}[ANOVA]{Analysis of Variance\acroextra{, a set of
  statistical techniques to identify sources of variability between groups}}
\acro{ATP}{Adenosine Triphosphate}
\acro{BCGSC}{British Columbia Genome Sciences Centre}
\acro{CIHR}{the Canadian Institute for Health Research}
\acro{DNA}{Deoxyribonucleic Acid}
\acro{GA}{Genome Analyzer}
\acro{kb}{kilobase}
\acro{ETC}{Electron Transport Chain}
\acro{numt}{nuclear-mitochondrial sequences}
\acro{G3}{Genomics Genetics and Gerontology}
\acro{FMNH}{Flavin Mononucleotide}
\acro{FoGS}[FoGS]{The Faculty of Graduate Studies}
\acro{MAF}{Minor Allele Frequency}
\acro{MELAS}{Mitochondrial Encephalopathy Lactic Acidosis and Stroke-like episodes}
\acro{MERRF}{Myoclonic Epilepsy and Ragged-Red Fibres}
\acro{mtDNA}{Mitochondrial DNA}
\acro{NADH}{Nicotinamide Adenine Dinucleotide}
\acro{PCR}{Polymerase Chain Reaction}
\acro{LHON}{Leber's Hereditary Optic Neuropathy}
\acro{rCRS}{revised Cambridge Reference Sequence}
\acro{RNA}{Ribonucleic Acid}
\acro{ROS}{Reactive Oxygen Species}
\acro{RR}{Relative Risk}
\acro{rRNA}{Ribosomal RNA}
\acro{tRNA}{Transfer RNA}
\acro{OR}{Odds Ratio}
\acro{SNP}{Single Nucleotide Polymorphism}
\acro{URL}{Unique Resource Locator\acroextra{, used to describe a
    means for obtaining some resource on the world wide web}}
\end{acronym}

% You can also use \newacro{}{} to only define acronyms
% but without explictly creating a glossary
% 
% \newacro{ANOVA}[ANOVA]{Analysis of Variance\acroextra{, a set of
%   statistical techniques to identify sources of variability between groups.}}
% \newacro{API}[API]{application programming interface}
% \newacro{GOMS}[GOMS]{Goals, Operators, Methods, and Selection\acroextra{,
%   a framework for usability analysis.}}
% \newacro{TLX}[TLX]{Task Load Index\acroextra{, an instrument for gauging
%   the subjective mental workload experienced by a human in performing
%   a task.}}
% \newacro{UI}[UI]{user interface}
% \newacro{UML}[UML]{Unified Modelling Language}
% \newacro{W3C}[W3C]{World Wide Web Consortium}
% \newacro{XML}[XML]{Extensible Markup Language}
        % always input, since other macros may rely on it
 
\textspacing            % begin one-half or double spacing
 
%   10. Acknowledgements (optional)
% \include{ack}
 
%   11. Dedication (optional)
 
% Body of Thesis (not all sections may apply)
\mainmatter
 
\acresetall     % reset all acronyms used so far
 
%    1. Introduction
\chapter{Introduction}\label{intro}

\section{Aging as a Genetic Disease}

\subsection{The ``Healthy Aging'' Phenotype}
Our goal is to study biological mechanisms of aging by identifying genetic variants that are associated with healthy aging.  This study focuses on individuals who have reached the upper end of the normal human lifespan in good health, as opposed to other longevity-based studies that focus on centenarians who may not be exceptionally healthy\cite{Costa2009,Sebastiani2012,zhang2003strikingly}.

This project has been carried out using samples and phenotype data from the Genomics, Genetics and Gerontology ($G^{3}$) Study of Healthy Aging. In this study, cases are defined as having a ``healthy aging'' phenotype if they reached the age of 85 years without being diagnosed with cancer, (excluding non-melanoma skin cancer) cardiovascular disease, major pulmonary disease (excluding asthma), Alzheimer disease or diabetes.  They have been further characterized by means of the Mini Mental State Examination for determination of moderate to severe cognitive impairment\cite{folstein1975mini}, the Timed Up and Go test of basic mobility skills\cite{podsiadlo1991timed} , the Geriatric Depression Scale\cite{yesavage1983geriatric} and the Instrumental Activities of Daily Living Scale\cite{katz1983assessing}.

Controls are between the ages of 40 and 54 years, and were not recruited with respect to health status.  As such, they are representative of the general population with respect to their probability of reaching the age of 85 years without acquiring one of the five common age-related diseases listed above.  Ideally, our controls would be a random sample of the population at the time that our cases were in mid-life, and we believe that these controls are a good proxy for that ideal sample.  Specifically, we believe that the allele frequencies of our control sample should be a good approximation of the allele frequencies of the (now largely deceased) population that our cases originated from.

\section{Mitochondria and Aging}

\subsection{Mitochondrial Genome Structure and Regulation}
Human mitochondria have a 16.5 kb circular genome, which encodes 13 protein-coding genes, (See table ~\ref{intro.table.mito_proteins}) 22 \ac{tRNA}s, and two \ac{rRNA}s (see figure~\ref{intro.fig.mtdna_map}).  The protein-coding genes encode subunits of the mitochondrial electron transport chain complexes I, III and IV, and two subunits of ATP synthase.  In contrast with the nuclear genome, there is very little non-coding sequence in the human mitochondrial genome.  The majority of the non-coding sequence is contained within the 1.1 kb control region, where three known promoters coordinate expression of the entire mitochondrial chromosome.  Outside of the control region, the mitochondrial genes are tightly spaced, with clusters of tRNA genes located between protein-coding genes.

\begin{figure}
\noindent\makebox[\textwidth]{%
  \includegraphics[width=1.4\textwidth]{fig/mtdna_map.png}}
  \caption[Map of the Human Mitochondrial Genome]{
    \small{\textbf{Map of the Human Mitochondrial Genome.} Non-coding control region (position 16,024\--576) is shown in grey.  Protein-coding genes are shown in blue, while RNA-coding genes are shown in red.  All gene labels are from the HUGO Gene Nomenclature Committee (www.genenames.org)}}
  \label{intro.fig.mtdna_map}
\end{figure}

\begin{table}[htbp]
\begin{minipage}{\textwidth}
\caption{Selected Mitochondrial Diseases}
\label{intro.table.mito_disease}
\noindent\makebox[\textwidth]{%
\footnotesize
\begin{tabular}{r l r l}
OMIM ID & Name                   & rCRS Positions Mutated  & Symptoms                  \\ \hline
 535000 & LHON                   & 11,778, 3,460, 14484    & Blindness                 \\
 540000 & MELAS                  & 3,243,                  & Myopathy, Lactic acidosis \\
 220110 & Complex IV Deficiency  & (various mutations in MT-CO1\--3) & Myopathy         \\
 256000 & Leigh Syndrome         & 4,681                   & CNS Lesions               \\
 545000 & MERRF                  & 8,344                   & Seizures, myopathy        \\
 530000 & Kearns-Sayre Syndrome  & (various deletions)     & Blindness, cardiomyopathy \\
 157640 & CPEO                   & (various deletions)     & Eye turn, hypogonadism
\end{tabular}}
\end{minipage}
\end{table}

Hundreds of additional mitochondrial proteins are encoded by the nuclear genome, and coordinated control of the two genomes is required for normal mitochondrial function\cite{hock2009transcriptional,ryan2007mitochondrial}.  Mitochondrial gene expression is controlled by transcription factors (TFAM, TFB1M, TFB2M) and an RNA polymerase (POLRMT) that are encoded on the nuclear genome.  A transcriptional regulatory network links a master regulator, PGC-1$\alpha$ (also known as PPARGC1A) to the mitochondrial genome.

\begin{table}[htbp]
\begin{minipage}{\textwidth}
\caption{Mitochondrial Protein Genes}
\label{intro.table.mito_proteins}
\noindent\makebox[\textwidth]{%
\footnotesize
\begin{tabular}{l r l r}
Gene & Uniprot Acession & ETC Complex & rCRS Position \\ \hline
MT-ND1  &           P03886 & Complex I   &   3,307--4,262 \\
MT-ND2  &           P03891 & Complex I   &   4,470--5,511 \\
MT-COX1 &           P00395 & Complex IV  &   5,904--7,445 \\ 
MT-COX2 &           P00403 & Complex IV  &   7,586--8,269 \\
MT-ATP8 &           P03928 & Complex V   &   8,366--8,572 \\
MT-ATP6 &           P00846 & Complex V   &   8,527--9,207 \\
MT-COX3 &           P00414 & Complex IV  &   9,207--9,990 \\ 
MT-ND3  &           P03897 & Complex I   & 10,059--10,404 \\
MT-ND4L &           P03901 & Complex I   & 10,470--10,766 \\
MT-ND4  &           P03905 & Complex I   & 10,760--12,137 \\
MT-ND5  &           P03915 & Complex I   & 12,337--14,148 \\
MT-ND6  &           P03923 & Complex I   & 14,149--14,673 \\
MT-CYB  &           P00156 & Complex III & 14,747--15,887 \\
\end{tabular}}
\end{minipage}
\end{table}

The mitochondrial genome also contains a 1.1 kb control region (position 16024-576 on GenBank NC\_012920) which includes promoters for both the heavy and light strands, and the heavy strand origin of replication.  The control region also contains numerous transcription factor binding sites.  There are three hyper-variable sequences (HVS1, HVS2 and HVS3) within the control region that contain a relatively high density of polymorphisms, in comparison to the rest of the mitochondrial genome\cite{Stoneking2000}.

The human mitochondrial genome is inherited exclusively from the mother.  Paternal mitochondria are selectively degraded after fertilization, by ubiquitin-mediated proteasomal degradation\cite{sutovsky2003early,thompson2003ubiquitination}.  There is no conclusive evidence for recombination in human \ac{mtDNA}\cite{eyre2001does}.  An extensive map of the geographic distribution of mitochondrial haplogroups in human populations has been recorded.  Together with geographic and genotype data from the non-recombining portion of the Y chromosome, this information has helped to trace early human migration out of Africa and across the globe\cite{behar2007genographic}.

The mitochondrial genome is also highly polymorphic in all human populations.  A previous study of European mitochondrial genome diversity identified 144 single nucleotide polymorphisms present in $>1\%$ of a sample of 928 publicly available European mitochondrial genome sequences \cite{saxena2006comprehensive}.

Numerous mitochondrial genetic diseases have been identified\cite{chinnery1999mitochondrial,schapira2012mitochondrial}. Several of these diseases, with their characteristic mutations are listed in table ~\ref{intro.table.mito_disease}. Symptoms vary widely, and include blindness, deafness, diabetes and ataxia.

\subsection{Reactive Oxygen Species}
Mitochondria are thought to contribute to the aging process through the production of \ac{ROS}, as a byproduct of oxidative phosphorylation\cite{Wallace1999, Moreno-Loshuertos2006}. Prolonged exposure to intracellular \ac{ROS} can cause damage to protein and lipids, and can cause somatic mutations in both the nuclear and mitochondrial genomes.

During oxidative phosphorylation, electrons are passed from reduced \ac{NADH} and \ac{FMNH} to a group of mitochondrial inner membrane-bound enzymes that comprise the electron transport chain.  Electrons are passed down the chain in a series of redox reactions, releasing energy that is used to pump protons into the intermembrane space.  These reactions maintain the mitochondrial elechemical gradient that drives the production of ATP.  The majority of electrons passing through the electron transport chain will finally be combined with \ce{H^{+}} and \ce{1/2O2} to form \ce{H2O}, but a small percentage will form side-reactions that result in the production of highly unstable superoxide radicals, \ce{O2^{-.}}.  Superoxide quickly reacts with \ce{H2O} to form hydrogen peroxide, (\ce{H2O2}) itself a strong oxidizing agent.  Although small amounts of ROS are a normal byproduct of cellular metabolism, the accumulated effects of these reactions can degrade tissue, cause somatic mutations and lead to cellular senescence\cite{Toyokuni1999, Passos2007}.

\subsection{The Role of Mitochondria in Apoptosis}
Mitochondria integrate several intracellular signals including DNA damage response and pro-survival signals, as well as metabolic signals such as the ADP/ATP ratio and intracellular \ce{Ca^2+} concentrations.  Under high cellular stress conditions, these signals can initiate cell death via the intrinsic apoptotic pathway.  The pro-apoptotic proteins BAX and BAK are recruited to the mitochondrial membrane, resulting in increased membrane permeability and release of Cytochrome-c and SMAC/DIABLO from the mitochondrial intermembrane space into the cytosol.  The release of Cytochrome-c and SMAC/DIABLO leads to the activation of effector caspases that initiate the process of apoptosis.

Apoptosis is a key protective mechanism against cancer.  When a cell acquires mutations or DNA damage that may lead to escape from the cell cycle and uncontrolled cell division, the apoptotic pathway can be activated to prevent the development of a malignancy.  Model organisms such as p53 knockout mice fail to activate the intrinsic apoptotic pathway in response to DNA damage and develop malignancies at a much higher rate than wild-type mice\cite{symonds1994p53}.  There is also evidence that variation in the mitochondrial genome itself can alter the probability that a cell will undergo apoptosis.  Studies of a lymphoblastoid cell line showed that a A4263G mutation in the mitochondrial isoleucine tRNA could alter mitochondrial membrane potential and lead to an increased rate of apoptosis\cite{yuqi2009voltage}.  Some have argued that many of the phenotypic hallmarks of aging (muscle loss, wrinkled skin, functional decline of internal organs) are due to the accumulated effects of apoptosis and senescence\cite{campisi2005senescent}.  They hypothesize that successful aging, (defined as reaching the age of 85 without being diagnosed with cancer, cardiovascular disease, diabetes, major pulmonary disease, or Alzheimer disease.)\cite{halaschek2009genetic} requires a fine balance between cancer surveillance by apoptosis and a maintenance of healthy pre-senescent tissue\cite{rodier2007two}.

\subsection{The Role of Mitochondria in Cellular Senescence}
Several lines of evidence indicate that mitochondria play a role in induction of cellular senescence. Senescent cells are characterized by growth arrest in the G1 phase of the cell cycle, accumulation of H2A.X foci and increased p53 activity indicative of DNA damage, and decreased telomere length\cite{passos2005mitochondria}.  The telomerase reverse transcriptase hTERT is translocated to mitochondria in response to oxidative stress, where it increases the rate of mtDNA damage and promotes apoptosis\cite{santos2004mitochondrial}.  This relationship between telomere maintenance and mtDNA maintenance is a recent discovery, and is not yet completely understood\cite{passos2007dna}.  Cells grown in high oxygen concentrations become senescent at an increased rate, and senescence can be delayed by addition of antioxidants or mild uncoupling agents to the growth medium\cite{haendeler2004antioxidants}.

\subsection{Somatic Mitochondrial DNA Mutations and Aging}
Mutations in the mitochondrial genome accumulate with age in somatic tissues. Mitochondrial DNA mutations have been observed to correlate with age in tissues such as heart muscle,\cite{hayakawa1992age} brain,\cite{soong1992mosaicism} and skeletal muscle\cite{melov1995marked}.  In addition to point mutations, accumulation of \ac{ROS}-damaged deoxyguanosine in the form of 8-Hydroxy-deoxyguanosine has been observed.

\subsection{Mitochondrial Heteroplasmy and Tissue Heterogeneity}
The number of mitochondria per cell varies from zero in red blood cells to several hundred in skeletal muscle cells, and each mitochondrion contains several copies of the mitochondrial genome.  Mutations can arise in somatic cells because of oxidative damage or replication errors by DNA polymerase-$\gamma$, and can be propagated to daughter cells after division.  Since each cell contains many copies of the mitochondrial genome, there may be a combination of mtDNA alleles in a particular cell or tissue\cite{Gyllensten2000,Li2010a}.  This phenomenon is known as heteroplasmy.  Several mitochondrial diseases, such as \ac{MERRF} or \ac{MELAS}, do not present physiological symptoms unless the causative mutation accumulates beyond a certain threshold level, sufficient to disrupt normal mitochondrial function\cite{Rossignol2003}.

Although the accumulation of somatic mtDNA mutations is suspected to play a role in the aging process, our study is designed to detect heritable genetic factors that influence long-term good health.  Mutations that arise in skeletal muscle, epithelium, neurons and other somatic tissues are not passed on in the germline.  Only mutations that arise in the ova (or pre-oval germ cell lineage) can be passed on to the next generation.

\subsection[Reported mtDNA Associations]{Reported Associations of Mitochondrial Genome Variants with Longevity}
Several longevity-associated mtDNA variants have been reported in populations around the world.  A control region polymorphism at position 150 was associated with longevity in the Italian population and has been hypothesized to cause a re-organization of an origin of replication on the mtDNA\cite{zhang2003strikingly}.  The comparison of 52 centenarians (age range 99\--106 years) and 117 controls (age range 18\--98 years) showed a statistically significant difference (\ac{OR} $=5.09$, $P=0.0035$, Fisher's exact test) in the frequency of homoplasmic C150T transition in leukocytes.  Furthermore, the researchers noted that the abundance of heteroplasmic C150T mutation in fibroblasts was correlated with age.  

The association signal at control region position 150 has been replicated in both the Japanese and Finnish populations\cite{niemi2005combination}.  In Finns, a comparison of 46 seniors (age 90 or 91 years) and 57 middle-aged controls showed a significant association (\ac{OR}$=1.50$, $P=0.037$, $\chi^{2}$ test) of the 150T allele with longevity.  A similar result was found in a smaller Japanese sample set of 19 seniors and 9 controls (\ac{OR}$=1.41$, $P=0.032$, $\chi^{2}$ test).

A polymorphism in the MT-ND2 gene at position 5,178 of the coding region was found to be associated with longevity in the Japanese population\cite{Tanaka1998}.  The study investigated the relative frequencies of the 5178A and 5178C alleles, and found that the 5178A allele in 9 of 11 centenarians, versus 12 of 43 controls.  This same polymorphism was also associated with glucose tolerance in Japanese men, and may contribute to resistance to type II diabetes.  The MT-ND2 gene encodes a subunit of \ac{NADH} dehydrogenase, complex I of the mitochondrial electron transport chain.

\section{Hypothesis and Specific Aims}
We hypothesize that healthy aging is influenced by sequence variation in the mitochondrial genome.  Therefore, one or more common mitochondrial alleles will be associated with healthy aging.

The specific aims of this study are as follows:

\begin{enumerate}
\item Survey the mitochondrial genomes of cases and controls for sequence variants.
\item Determine whether common variation in the mitochondrial genome is associated with healthy aging in our study population.
\end{enumerate}

% Force a new page
\newpage

 
%    2. Main body
% Generally recommended to put each chapter into a separate file
\chapter{Variant Detection}\label{var}
\section{Introduction}\label{var.introduction}
The mitochondrial genome is amenable to targeted resequencing by second-generation technologies.  Its relatively small size (16.5 \ac{kb}) and low repetitive sequence content make it much more likely that a unique sequence alignment can be determined for the short reads generated by current second-generation sequencing systems.

In order to identify the extent of mitochondrial genome variation in our sample set, we used two sequencing technologies for two distinct segments of the mitochondrial genome.  For the non-coding control region (position 16,024\--576), we performed bi-directional Sanger sequencing on all 419 cases and 415 controls.  A pooled Illumina Genome Analyzer (GA) sequencing strategy was used to identify variants in the entire mitochondrial genome.  See figure ~\ref{var.fig.design} for an outline of the variant detection strategy.

\begin{figure}
  \begin{center}
    \includegraphics[width=\textwidth]{fig/design.pdf}
  \end{center}
  \caption[Experimental Design for Variant Detection]{
    \small{\textbf{Experimental Design for Variant Detection.} Two sequencing technologies were used to identify mitochondrial genome variation.  The control-region was PCR-amplified in two segments and sequenced by Sanger sequencing in individuals.  The entire mitochondrial genome was PCR-amplified, pooled, and sequenced on the Illumina Genome Analyzer.  Coding region variants were carried forward to Sequenom genotyping in individuals.}}
  \label{var.fig.design}
\end{figure}

\section{Methods}\label{var.methods}
This study was approved by the joint Clinical Research Ethics Board of the British Columbia Cancer Agency and the University of British Columbia. All subjects gave written informed consent.

\subsection{Subjects and Samples}\label{methods.samples}
The subjects of this study were 419 healthy elderly individuals (cases) and 415 mid-life controls.  Cases were $> 85$ years old at the time of recruitment, and had not been diagnosed with cancer, cardiovascular disease, Alzheimer disease or diabetes.  Controls were 40-50 years old at recruitment, and were ascertained without regard to health status.  All participants are of European descent, based on subject-reported ethnicity of their four grandparents.  Total DNA was extracted from peripheral blood leukocytes using the Gentra Puregene Blood Kit (Qiagen), according to the manufacturer's protocol.

\subsection{Control Region PCR and Sanger Sequencing}\label{var.methods.control_region}
PCR primers were designed not to overlap with common polymorphic loci. In-silico PCR was performed using web service based at Kyushu University, to ensure that no nuclear DNA segments would be co-amplified\cite{kyushu}.  The mitochondrial control region was PCR-amplified with Platinum Pfx polymerase (Invitrogen).  PCR reactions were performed in 20 $\mu$L total volume containing: 20 ng template genomic DNA, 10 $\mu$M each of forward primer (MAP001\_F or MAP002.1\_F) and reverse primer (MAP001\_R or MAP002.1\_R) (Table ~\ref{var.table.primers}), 0.4 U Platinum Pfx enzyme, 10 mM each dNTPs, and 1x Phusion Buffer GC.  Forward and reverse primers incorporated the -21M13F (TGTAAAACGACGGCCAGT) and M13R (CAGGAAACAGCTATGAC) extensions, respectively, at their 5' ends.  Sequencing reactions were carried out as described previously \cite{brooks2004germline}.

\subsection{Sanger Sequence Assembly}
Sanger sequence traces were aligned to the \ac{rCRS} reference sequence (GenBank accession NC\_012920) with the Phred/Phrap/Consed suite, version 20.0 \cite{gordon1998consed,ewing1998base1,ewing1998base2}.  Polymorphisms were first detected automatically using Polyphred version 6.18. To minimize false-positives all non-reference alleles were manually confirmed by visual inspection of chromatograms by two people.

\subsection{Long PCR}\label{var.methods.long_pcr}
The mitochondrial genome was amplified using long-PCR with Phusion polymerase (Finnzymes).  PCR reactions were performed in 20 $\mu$L total volume containing: 20 ng template genomic DNA (2 ng/$\mu$L), 10 $\mu$M each of forward primer MAP011.1\_F and reverse primer MAP011.1\_R (Table ~\ref{var.table.primers}), 0.4 U Phusion enzyme, 10mM each dNTPs, and 1x Phusion Buffer GC.  The thermocycler program was: 1.) initial melt at 98\textdegree C for 30 seconds, 2.) melt at 98\textdegree C for 10 seconds, 3.) anneal/extend at  72\textdegree C for 8 minutes, 15 seconds 4.) repeat steps 2 and 3, 29 times 5.) final extension at 72\textdegree C for 10 minutes.

\begin{table}[htbp]
\begin{minipage}{\textwidth}
\caption[List of PCR Primers]{PCR primers used for Sanger sequencing and long-PCR.}
\label{var.table.primers}
\noindent\makebox[\textwidth]{%
\footnotesize
\begin{tabular}{l c l l}
Primer ID  & T$_{m}$ (\textdegree C) & Sequence & rCRS Position\\ \hline
MAP011.1-F & 66.3 & GGGAGCTCTCCATGCATTTGG      &     34-54 \\
MAP011.1-R & 64.7 & AGACCTGTGATCCATCGTGATGTC   & 16,558-12\\
MAP001-F   & 57.1 & (-21M13-Fwd\footnote{`-21M13-Fwd' = TGTAAAACGACGGCCAGT})GAAAAAGTCTTTAACTCCACCATT & 15,961-15,984\\
MAP001-R   & 58.9 & (M13-Rev\footnote{`M13-Rev' = CAGGAAACAGCTATGAC})TACTGCGACATAGGGTGCTC & 107-126\\
MAP002.1-F & 59.3 & (-21M13-Fwd)GAGCTCTCCATGCATTTGG &   36-54\\
MAP002.1-R & 57.3 & (M13-Rev)AGGGTGAACTCACTGGAACG   & 707-726\\
\end{tabular}}
\end{minipage}
\end{table}

\subsection{Construction of DNA Pools}\label{var.methods.pools}
DNA products from long-PCR were quantitated with Quant-iT\texttrademark PicoGreen\textsuperscript{\textregistered} reagent (Invitrogen).  Two DNA pools were constructed. One pool consisted of 10 ng mtDNA from each of 419 case samples, and the other consisted of 10 ng mtDNA from each of 415 control samples.  DNA was concentrated by speed-vac.

\subsection{Library Construction and Sequencing}\label{var_detect.methods.library}
Library construction and DNA sequencing was carried out by the sequencing platform of the BC Genome Sciences Centre.  Pooled mtDNA was sheared using sonication and size-separated using electrophoresis. The $\sim300$-bp fraction was isolated for library construction using the Illumina Genome Analyzer single-end library protocol (Illumina). Sequencing was performed on an Illumina GA using two lanes of a flow cell per pool, generating 36-bp reads.

\begin{table}[htbp]
\begin{minipage}{\textwidth}
\caption{Summary of Illumina Sequence Mapping (Untrimmed Reads)}
\label{var.table.illumina.untrimmed}
\noindent\makebox[\textwidth]{%
\footnotesize
\begin{tabular}{l r r l r l}
Chr.           & Length      & \multicolumn{2}{c}{Reads Mapped, Cases} & \multicolumn{2}{c}{Reads Mapped, Controls} \\ \hline 
 1             & 249,250,621 &    512,872 &   (4.30\%) &    539,464 &   (4.23\%) \\ 
 2             & 243,199,373 &     81,462 &   (0.68\%) &     85,478 &   (0.67\%) \\ 
 3             & 198,022,430 &     74,357 &   (0.62\%) &     77,525 &   (0.61\%) \\ 
 4             & 191,154,276 &     48,754 &   (0.41\%) &     53,598 &   (0.42\%) \\ 
 5             & 180,915,260 &    206,336 &   (1.73\%) &    225,336 &   (1.77\%) \\ 
 6             & 171,115,067 &     32,858 &   (0.28\%) &     35,324 &   (0.28\%) \\ 
 7             & 159,138,663 &    170,779 &   (1.43\%) &    227,925 &   (1.79\%) \\ 
 8             & 146,364,022 &     35,068 &   (0.29\%) &     38,300 &   (0.30\%) \\ 
 9             & 141,213,431 &     27,073 &   (0.23\%) &     26,942 &   (0.21\%) \\ 
10             & 135,534,747 &     25,878 &   (0.22\%) &     25,809 &   (0.20\%) \\ 
11             & 135,006,516 &     97,174 &   (0.81\%) &     98,442 &   (0.77\%) \\ 
12             & 133,851,895 &     28,901 &   (0.24\%) &     30,747 &   (0.24\%) \\ 
13             & 115,169,878 &     35,384 &   (0.30\%) &     36,679 &   (0.29\%) \\ 
14             & 107,349,540 &     23,239 &   (0.19\%) &     24,482 &   (0.19\%) \\ 
15             & 102,531,392 &     13,155 &   (0.11\%) &     14,121 &   (0.11\%) \\ 
16             &  90,354,753 &     13,580 &   (0.11\%) &     13,829 &   (0.11\%) \\ 
17             &  81,195,210 &    311,656 &   (2.61\%) &    299,374 &   (2.35\%) \\ 
18             &  78,077,248 &     19,447 &   (0.16\%) &     20,532 &   (0.16\%) \\ 
19             &  59,128,983 &      7,709 &   (0.06\%) &      7,346 &   (0.06\%) \\ 
20             &  63,025,520 &     11,142 &   (0.09\%) &     11,389 &   (0.09\%) \\ 
21             &  48,129,895 &     13,807 &   (0.12\%) &     14,062 &   (0.11\%) \\ 
22             &  51,304,566 &      5,967 &   (0.05\%) &      5,941 &   (0.05\%) \\ 
 X             & 155,270,560 &     54,631 &   (0.46\%) &     62,053 &   (0.49\%) \\ 
 Y             &  59,373,566 &     11,447 &   (0.10\%) &     11,022 &   (0.09\%) \\ 
MT             &      16,569 &  4,286,809 &  (35.94\%) &  4,681,659 &  (36.68\%) \\ 
other          &   6,110,758 &      5,068 &   (0.04\%) &      5,087 &   (0.04\%) \\ \hline
total mapped   & -           &  6,154,553 &  (51.59\%) &  6,672,466 &  (52.27\%) \\ 
total unmapped & -           &  5,774,653 &  (48.41\%) &  6,092,354 &  (47.73\%) \\ 
grand total    & -           & 11,929,206 & (100.00\%) & 12,764,820 & (100.00\%) \\
\end{tabular}}
\end{minipage}
\end{table}

\begin{table}[htbp]
\begin{minipage}{\textwidth}
\caption{Summary of Illumina Sequence Mapping (Trimmed Reads)}
\label{var.table.illumina.trimmed}
\noindent\makebox[\textwidth]{%
\footnotesize
\begin{tabular}{l r r l r l}
Chr. &      Length & \multicolumn{2}{c}{Reads Mapped, Cases} & \multicolumn{2}{c}{Reads Mapped, Controls} \\ \hline
 1 &   249,250,621 &   496,262 &   (6.92\%) &     508,839 &   (6.92\%) \\ 
 2 &   243,199,373 &    95,787 &   (1.34\%) &      94,431 &   (1.28\%) \\ 
 3 &   198,022,430 &    86,431 &   (1.20\%) &      82,319 &   (1.12\%) \\ 
 4 &   191,154,276 &    64,637 &   (0.90\%) &      65,229 &   (0.89\%) \\ 
 5 &   180,915,260 &   263,349 &   (3.67\%) &     270,725 &   (3.68\%) \\ 
 6 &   171,115,067 &    38,931 &   (0.54\%) &      37,264 &   (0.51\%) \\ 
 7 &   159,138,663 &   112,712 &   (1.57\%) &     107,757 &   (1.47\%) \\ 
 8 &   146,364,022 &    43,377 &   (0.60\%) &      42,340 &   (0.58\%) \\ 
 9 &   141,213,431 &    35,488 &   (0.49\%) &      32,654 &   (0.44\%) \\ 
10 &   135,534,747 &    32,305 &   (0.45\%) &      29,541 &   (0.40\%) \\ 
11 &   135,006,516 &   101,180 &   (1.41\%) &      96,423 &   (1.31\%) \\ 
12 &   133,851,895 &    35,422 &   (0.49\%) &      35,238 &   (0.48\%) \\ 
13 &   115,169,878 &    36,647 &   (0.51\%) &      36,389 &   (0.50\%) \\ 
14 &   107,349,540 &    29,971 &   (0.42\%) &      28,723 &   (0.39\%) \\ 
15 &   102,531,392 &    16,489 &   (0.23\%) &      15,896 &   (0.22\%) \\ 
16 &    90,354,753 &    17,579 &   (0.25\%) &      16,484 &   (0.22\%) \\ 
17 &    81,195,210 &   392,551 &   (5.47\%) &     340,912 &   (4.64\%) \\ 
18 &    78,077,248 &    22,205 &   (0.31\%) &      21,735 &   (0.30\%) \\ 
19 &    59,128,983 &     8,908 &   (0.12\%) &       7,365 &   (0.10\%) \\ 
20 &    63,025,520 &    14,805 &   (0.21\%) &      13,690 &   (0.19\%) \\ 
21 &    48,129,895 &    16,454 &   (0.23\%) &      14,409 &   (0.20\%) \\ 
22 &    51,304,566 &     7,768 &   (0.11\%) &       6,995 &   (0.10\%) \\ 
 X &   155,270,560 &    66,354 &   (0.93\%) &      64,348 &   (0.88\%) \\ 
 Y &    59,373,566 &    11,477 &   (0.16\%) &      10,403 &   (0.14\%) \\ 
MT &        16,569 & 3,750,715 &  (52.29\%) &   4,015,230 &  (54.64\%) \\ 
other &  6,110,758 &     5,121 &   (0.07\%) &       4,722 &   (0.06\%) \\ \hline 
  total mapped & - & 5,802,925 &  (80.90\%) &   6,000,061 &  (81.64\%) \\ 
total unmapped & - & 1,370,272 &  (19.10\%) &   1,348,969 &  (18.36\%) \\ 
   grand total & - & 7,173,197 & (100.00\%) &   7,349,030 & (100.00\%) \\
\end{tabular}}
\end{minipage}
\end{table}

\subsection{Statistical Analysis}\label{var.methods.statistical_analysis}
In order to assess the effect of read trimming on mapping, alignments were done with both full 36-base reads and trimmed reads.  For trimmed reads, the BWA read-trimming parameter (q=25) was used.  Short sequence reads were aligned to the GRCh37 (hg19) reference using the BWA sequence alignment program, version 0.6.1-r104\cite{li2009fast}.  Aside from the read-trimming parameter, all reads were mapped using default BWA parameters.

Per-base quality scores for both untrimmed and trimmed reads were calculated with FastQC software\cite{andrews2010fastqc} (See figures ~\ref{var.fig.base_quality_case}, ~\ref{var.fig.base_quality_cont})

SNPs were detected by analyzing BWA `pileup' output files with a custom perl script. At each position, the numbers of reference and non-reference bases were counted.  Only those bases with phred-scaled quality scores of 40 were included for SNP detection.

\section{Results}\label{var.results}

\subsection{Sequencing of the Mitochondrial Conrol Region}

The highly polymorphic mitochondrial control region rCRS (position 16024-576) was sequenced using bi-directional Sanger sequencing.  We discovered 277 SNPs in the control region that were present in at least one sample.

\begin{figure}
  \begin{center}
    \subfloat[Alignment]{\includegraphics[width=0.85\textwidth]{fig/consed.png}}\\
    \subfloat[Traces]{\includegraphics[width=0.85\textwidth]{fig/consed_trace.png}}  \end{center}
  \caption[Screenshots for Variant Discovery with Phred/Phrap/Consed]{
    \small{\textbf{SNP Calling by Phred/Phrap/Consed + PolyPhred.} (\textbf{a}) Reads were aligned to the revised Cambridge Reference Sequence (\ac{rCRS}).  Each sample was sequenced in both forward and reverse directions. Only a subset of samples are shown Sample IDs are at left in yellow type.  (\textbf{b}) Variants were identified automatically using PolyPhred, and confirmed manually by visual inspection of sequence traces. Two samples (127\_WIL and 128\_SIN) with differing alleles at contig position 1,288 (rCRS position 16,288) are shown.}}
  \label{var.fig.consed}
\end{figure}

\begin{figure}[htpb]
  \begin{center}
    \subfloat[a]{\includegraphics[width=\textwidth]{fig/heteroplasmy_01.png}}\\
    \subfloat[b]{\includegraphics[width=\textwidth]{fig/heteroplasmy_02.png}}  \end{center}
  \caption[Putative Heteroplasmic Positions]{
    \small{\textbf{Putative Heteroplasmic Positions.} Heteroplasmy was observed in some samples by identifying double-peaks in sequence traces. (\textbf{a}) Sample `157\_EPP' shows putative heteroplasmy level of $\sim 25\%$ at contig position 1,189 (rCRS position 16,189). (\textbf{b}) Sample `489\_SAM' shows putative heteroplasmy level of $\sim 50\%$ at contig position 1,126 (rCRS position 16,126).} Note that the relative heights of the two peaks at the heteroplasmic positions are consistent in forward and reverse reads.}
  \label{var.fig.heteroplasmy}
\end{figure}

\subsection{Next-Generation Sequencing of Pooled mtDNA}\label{var.results.illumina_sequencing}
The median depth of sequence coverage was 31,134 reads for the case pool and 12,683 reads for the control pool.  This represents approximately 30x coverage for each sample that is included in the pool.  To reduce the number of false-positive variant calls that are due to sequencing errors, we only considered high-quality bases (base-quality score $> 35$ and mapping-qualty score $> 20$) for SNP-calling.  We identified 90 SNPs in the case pool and 113 SNPs in the control pool with MAF $> 1\%$. 84 of these SNPs are common to both pools, with 6 SNPs only being observed in the case pool and 29 SNPs only being observed in the control pool. (see figs ~\ref{var.fig.snp_case} and ~\ref{var.fig.snp_cont}).

Comparison of minor allele frequencies for control region SNPs in Sanger and Illumina GA datasets is shown in Figures ~\ref{var.fig.comparison_case} and ~\ref{var.fig.comparison_cont}.  We found close correlation (Spearman's $r = 0.88$ in cases, Spearman's $r = 0.88$ in controls, $N = 277$).  We also observed that pooled Illumina GA sequencing produced consistently lower MAF estimates than Sanger sequencing of individual samples in this region.

\begin{figure}
  \begin{center}
  \subfloat[Case Pool, Untrimmed]{\includegraphics[width=0.75\textwidth]{fig/hs0297_case_per_base_quality_untrimmed.png}} \\
  \subfloat[Case Pool, Trimmed]{\includegraphics[width=0.75\textwidth]{fig/hs0297_case_per_base_quality_trimmed.png}}
  \end{center}
  \caption[Per-base Quality Distributions]{
    \small{\textbf{Effect of Read-trimming on Per-base Quality Distributions (Case Pool)} Average base quality score was calculated at each read position, across all reads.  For each position, red line indicates median quality score, yellow box indicates interquartile range (25\--75\%), upper and lower whiskers represent 90\% and 10\% quantiles, respectively, and blue line represents mean quality score. The upwards shift in average quality for trimmed reads indicates that poor-qualty sequence near the 3' end of reads has been removed in trimmed reads. }}
  \label{var.fig.base_quality_case}
\end{figure}

\begin{figure}
  \begin{center}
  \subfloat[Control Pool, Untrimmed]{\includegraphics[width=0.75\textwidth]{fig/hs0298_cont_per_base_quality_untrimmed.png}} \\
  \subfloat[Control Pool, Trimmed]{\includegraphics[width=0.75\textwidth]{fig/hs0298_cont_per_base_quality_trimmed.png}}
  \end{center}
  \caption[Per-base Quality Distributions]{
    \small{\textbf{Effect of Read-trimming on Per-base Quality Distributions (Control Pool)} Average base quality score was calculated at each read position, across all reads.  For each position, red line indicates median quality score, yellow box indicates interquartile range (25\--75\%), upper and lower whiskers represent 90\% and 10\% quantiles, respectively, and blue line represents mean quality score.  The upwards shift in average quality for trimmed reads indicates that poor-qualty sequence near the 3' end of reads has been removed in trimmed reads.}}
  \label{var.fig.base_quality_cont}
\end{figure}

\begin{figure}
\noindent\makebox[\textwidth]{%
  \includegraphics[width=1.4\textwidth]{fig/coverage_case.png}}
  \caption[Sequence Coverage for Case Pool]{
    \small{\textbf{Sequence coverage across the mitochondrial genome (Case Pool).} Blue line indicates high-quality (phred-scaled quality score = 40) sequence coverage. Graph lines every 10,000-fold depth.}}
  \label{var.fig.coverage_case}
\end{figure}

\begin{figure}
\noindent\makebox[\textwidth]{%
  \includegraphics[width=1.4\textwidth]{fig/coverage_cont.png}}
  \caption[Sequence Coverage for Control Pool]{
    \small{\textbf{Sequence coverage across the mitochondrial genome (Control Pool).} Blue line indicates high-quality (phred-scaled quality score = 40) sequence coverage. Graph lines every 10,000-fold depth.}}
  \label{var.fig.coverage_cont}
\end{figure}

\begin{figure}
\noindent\makebox[\textwidth]{%
  \includegraphics[width=1.4\textwidth]{fig/illumina_snps_case.png}}
  \caption[Minor Allele Frequencies for Case Pool]{
    \small{\textbf{Minor allele frequencies (Case Pool).} Locations and minor allele frequencies for all SNPs detected by Illumina GA sequencing.  Base identities are indicated as follows: A = Red, C = Blue, G = Orange, T = Green.  Heights of data bars indicate minor allele frequencies, scale bars every 10\% allele frequency.}}
  \label{var.fig.snp_case}
\end{figure}

\begin{figure}
\noindent\makebox[\textwidth]{%
  \includegraphics[width=1.4\textwidth]{fig/illumina_snps_cont.png}}
  \caption[Minor Allele Frequencies for Control Pool]{
    \small{\textbf{Minor allele frequencies (Control Pool).} Locations and minor allele frequencies for all SNPs detected by Illumina GA sequencing.  Base identities are indicated as follows: A = Red, C = Blue, G = Orange, T = Green.  Heights of data bars indicate minor allele frequencies, scale bars every 10\% allele frequency.}}
  \label{var.fig.snp_cont}
\end{figure}

\begin{figure}
\noindent\makebox[\textwidth]{%
  \includegraphics[width=\textwidth]{fig/solexa_sanger_maf_comparison_case.png}}
  \caption[MAF Comparison for Cases]{
    \small{\textbf{MAF comparison (Cases).}  Minor allele frequencies were determined by both Sanger sequencing and by pooled Illumina GA sequencing for 277 SNPs in the control region.  The Spearman's rank correlation between the two estimates is 0.88 in the case sample set, and 0.91 in controls.  Dashed line indicates slope = 1; the least-squares regression line is indicated by a solid line.}}
  \label{var.fig.comparison_case}
\end{figure}

\begin{figure}
\noindent\makebox[\textwidth]{%
  \includegraphics[width=\textwidth]{fig/solexa_sanger_maf_comparison_cont.png}}
  \caption[MAF Comparison for Controls]{
    \small{\textbf{MAF comparison (Controls).}  Minor allele frequencies were determined by both Sanger sequencing and by pooled Illumina GA sequencing for 277 SNPs in the control region.  The Spearman's rank correlation between the two estimates is 0.91 in control sample set.  Dashed line indicates slope = 1; the least-squares regression line is indicated by a solid line.}}
  \label{var.fig.comparison_cont}
\end{figure}

\begin{table}[htbp]
\begin{minipage}{\textwidth}
\caption[Mitochondrial Non-synonymous SNPs]{Functional Consequences for Non-synonymous SNPs}
\label{var.table.non-synon}
\noindent\makebox[\textwidth]{%
\footnotesize
\begin{tabular}{l r c c l l l}
ID           & Position & MAF (Seniors) & MAF(Controls) & Gene & Amino Acid Change & PolyPhen  \\ \hline
rs28357980   &   4,917  &         0.073 &         0.060 &  MT-ND2 & N [Asn] $\Rightarrow$ D [Asp] & 0.129 (benign) \\
rs28358886   &   8,697  &         0.078 &         0.053 & MT-ATP6 & M [Met] $\Rightarrow$ I [Ile] & 0.890 (possibly damaging) \\
rs9645429    &   9,055  &         0.070 &         0.051 & MT-ATP6 & A [Ala] $\Rightarrow$ T [Thr] & 0.845 (possibly damaging) \\
rs2853826    &  10,398  &         0.055 &         0.059 &  MT-ND3 & T [Thr] $\Rightarrow$ A [Ala] & 0.000 (benign) \\
rs28359178   &  13,708  &         0.041 &         0.028 &  MT-ND5 & A [Ala] $\Rightarrow$ T [Thr] & 0.000 (benign) \\
rs3135031    &  14,766  &         0.084 &         0.072 & MT-CYTB & T [Thr] $\Rightarrow$ I [Ile] & 0.000 (benign) \\
rs28357681   &  14,798  &         0.109 &         0.077 & MT-CYTB & F [Phe] $\Rightarrow$ L [Leu] & 0.000 (benign) \\
rs2853508    &  15,326  &         0.245 &         0.218 & MT-CYTB & T [Thr] $\Rightarrow$ A [Ala] & 0.000 (benign) \\
rs3088309    &  15,452  &         0.134 &         0.118 & MT-CYTB & L [Leu] $\Rightarrow$ I [Ile] & 0.029 (benign)      
\end{tabular}}
\end{minipage}
\end{table}

\begin{table}[htbp]
\begin{minipage}{\textwidth}
\caption[Number of Variants by Gene]{Number of Variants by Gene}
\label{var.table.num_variants_by_gene}
\noindent\makebox[\textwidth]{%
\footnotesize
\begin{tabular}{l r r r r r}
         &           & \multicolumn{2}{c}{Total Variants} & \multicolumn{2}{c}{Variants per kb}\\  
Gene     & Size (bp) & Cases & Controls & Cases & Controls\\ \hline
MT-TF    &    71 &  0 &  0 &  0.0 &  0.0 \\ 
MT-RNR1  &   954 &  7 & 10 &  7.3 & 10.5 \\ 
MT-TV    &    69 &  0 &  0 &  0.0 &  0.0 \\ 
MT-RNR2  & 1,559 & 17 & 16 & 10.9 & 10.3 \\ 
MT-TL1   &    75 &  0 &  0 &  0.0 &  0.0 \\ 
MT-ND1   &   956 &  8 &  7 &  8.4 &  7.3 \\ 
MT-TI    &    69 &  0 &  0 &  0.0 &  0.0 \\ 
MT-TQ    &    72 &  1 &  1 & 13.9 & 13.9 \\ 
MT-TM    &    68 &  0 &  0 &  0.0 &  0.0 \\ 
MT-ND2   & 1,042 & 12 & 19 & 11.5 & 18.2 \\ 
MT-TW    &    68 &  0 &  0 &  0.0 &  0.0 \\ 
MT-TA    &    69 &  1 &  2 & 14.5 & 29.0 \\ 
MT-TN    &    73 &  0 &  0 &  0.0 &  0.0 \\ 
MT-TC    &    66 &  0 &  1 &  0.0 & 15.2 \\ 
MT-TY    &    66 &  0 &  0 &  0.0 &  0.0 \\ 
MT-CO1   & 1,542 & 11 & 16 &  7.1 & 10.4 \\ 
MT-TS1   &    69 &  1 &  1 & 14.5 & 14.5 \\ 
MT-TD    &    68 &  0 &  0 &  0.0 &  0.0 \\ 
MT-CO2   &   684 &  3 &  4 &  4.4 &  5.8 \\ 
MT-TK    &    70 &  0 &  1 &  0.0 & 14.3 \\ 
MT-ATP8  &   207 &  3 &  3 & 14.5 & 14.5 \\ 
MT-ATP6  &   681 &  7 &  9 & 10.3 & 13.2 \\ 
MT-C03   &   784 & 10 &  9 & 12.8 & 11.5 \\ 
MT-TG    &    68 &  1 &  1 & 14.7 & 14.7 \\ 
MT-ND3   &   346 &  7 &  6 & 20.2 & 17.3 \\ 
MT-TR    &    65 &  1 &  1 & 15.4 & 15.4 \\ 
MT-ND4L  &   297 &  2 &  3 &  6.7 & 10.1 \\ 
MT-ND4   & 1,378 & 18 & 24 & 13.1 & 17.4 \\ 
MT-TH    &    69 &  0 &  0 &  0.0 &  0.0 \\ 
MT-TS2   &    59 &  0 &  0 &  0.0 &  0.0 \\ 
MT-TL2   &    71 &  1 &  2 & 14.1 & 28.2 \\ 
MT-ND5   & 1,812 & 28 & 27 & 15.5 & 14.9 \\ 
MT-ND6   &   525 & 10 &  6 & 19.0 & 11.4 \\ 
MT-TE    &    69 &  0 &  0 &  0.0 &  0.0 \\ 
MT-CYTB  & 1,141 & 20 & 22 & 17.5 & 19.3 \\ 
MT-TT    &    66 &  3 &  5 & 45.5 & 75.8 \\ 
MT-TP    &    68 &  0 &  0 &  0.0 &  0.0 \\ \hline
All Protein-coding & 11,395 & 139 & 155 & 12.2 & 13.6 \\ 
All RNA-coding     &  4,021 &  33 &  41 &  8.2 & 10.2 \\  
\end{tabular}}
\end{minipage}
\end{table}

For each gene in the mitochondrial genome, the number of variants observed at $\geq 1\%$ frequency were tabulated (Table \ref{var.table.num_variants_by_gene}). The most variable protein-coding gene is MT-ND3, with 20.2 variants per \ac{kb} in cases, and 17.3 variants per \ac{kb} in controls.  The most variable RNA gene is MT-TT, with 45.5 variants per \ac{kb} in cases and 75.5 variants per \ac{kb} in controls.

\section{Discussion}
We have shown here that it is possible to discover variants across the entire mitochondrial genome in over 400 samples in a single sequencing experiment. By combining long-PCR with second-generation sequencing technology, we were able to estimate the alllele frequencies of over 300 mitochondrial SNPs in our study population. This technique will be useful for rapidly surveying a large sample set for mitochondrial SNPs. Given its small size and high copy number per cell, mtDNA is a good candidate for pooled targeted resequencing efforts. The size of the mitochondrial chromosome (16.5 \ac{kb}) makes it amenable to long PCR. The whole mtDNA genome can be amplified in one reaction, which simplifies the DNA pooling process.  A similar variant detection has been employed by another group, using a pool size of 20 samples\cite{wang2011estimating}.

Figures \ref{var.fig.coverage_case} and \ref{var.fig.coverage_cont} show that the entire mitochondrial genome was sufficiently covered by mapped reads to perform variant detection.  There are strong peaks in coverage in both the case pool and control pool near position 200 within the control region.  We attribute this peak to excess PCR primers that were carried through into the sequencing reaction.

A previous report showed accurate determination of allele frequencies of pooled genomic DNA on the ABI SOLiD, Roche 454 and Illumina GA II platforms \cite{druley2009quantification}. Our estimation of MAF from Illumina sequencing of DNA pools correlates strongly with MAF calculated using genotypes determined using Sanger sequence data (Spearman's $r$ = 0.88); this correlation is close to the value of $r^{2} = 0.9637$ published by Druley et al\cite{druley2009quantification}. The most likely source of discrepancy between these two datasets is due to small differences in the quantity of DNA that each sample contributes to the DNA pool. 

In our analyses, MAF estimated from Illumina GA data is about 25\% lower than our measurement from Sanger sequencing. We suggest that this discrepancy may represent a bias against mapping of reads containing non-reference bases. We suggest that a read that contains a real non-reference base in the form of a SNP is less likely to align than a read that contains no non-reference SNPs, and that this probem will be increased in low-quality sequence data.  This phenomenon, referred to as `reference bias,' has been observed in previous studies of next-generation sequence data\cite{degner2009effect}.

The number of variants observed in at least 1\% of samples varied from 0 (MT-TY, MT-TF for example) to 27 (MT-ND5) (see table ~\ref{var.table.num_variants_by_gene}).  When normalized by the length of the gene, the most variable genes are MT-ND3 (20.2 variants/kb in cases, 17.3 variants/kb in controls) and MT-TT (45.5 variants/kb in cases, 75.8 variants/kb in controls).  Note, however, that the short length of the tRNA genes ($\sim 70$ bp) leads to a highly variable estimate of variants/kb.  Overall the distribution of variants was similar in protien-coding and RNA-coding genes at roughly 10 variants/kb.

Although our study was not designed to investigate the role that heteroplasmic variants play in the aging process, we did detect a small number of putative heteroplasmic variants by Sanger sequencing.  For low levels of heteroplasmy, (below $\sim 25\%$) it would be difficult to distinguish a true heteroplasmic variant from background noise in the sequence trace.  The few instances of heteroplasmy that we were able to identify with some certainty appeared to be close to 50\% heteroplasmic (See ~\ref{var.fig.heteroplasmy} for a representative example).

% Force a new page
\newpage

\chapter{A Case-Control Association Study for Mitochondrial Variants and Healthy Aging}\label{assoc}
In order to identify variants that are associated with the healthy-aging phenotype, case-control association tests were performed using PLINK software\cite{Purcell2007}.  Each SNP is analyzed by comparing the major and minor allele frequencies in cases versus controls, by applying a Chi-squared ($\chi^{2}$) test.

The power of a Chi-squared test to detect a genetic association is based on a comparison of a null $\chi^{2}_{(1-\alpha)}$ distribution to an alternative $\chi^{2}$ distribution with non-centrality parameter $\lambda$, proportional to the effect size\cite{DeBakker2005}.  It is expressed as follows:

\begin{equation}
\mathrm{Power} = \mathrm{P}(\chi^{2} (df, \lambda) \geq \chi^{2}_{1 - \alpha} (df)),
\end{equation}

\noindent where:

\begin{equation}
\lambda = \Delta^{2} N = \left( \frac{(p - q)^{2}}{q} \right) N
\end{equation}

\noindent and for a $2 \times 2$ contingency table, the number of degrees of freedom ($df$) are one.  

Coding-region variants were nominated for genotyping based on three criteria.  Variants that showed a suggestive $P$-value ($< 0.05$) based on a comparison of the estimated \ac{MAF} from pooled Illumina GA II sequencing were genotyped, as were a set of 64 tag \ac{SNP}s that were designed to capture all common variants present at $>1\%$ in the European population, with linkage disequilibrium of at least $r^{2} = 0.8$\cite{saxena2006comprehensive}.  Finally, any variant with an estimated \ac{MAF} of $> 0.05$ based on pooled sequencing that did not fit the first two criteria was also included.  In total, 92 SNPs were nominated for genotyping (see Supplemental Table ~\ref{sup.tab.mito_snp_select}). 
 
Due to our limited statistical power to detect moderate effects in low-frequency variants, a \ac{MAF} cut-off of $10\%$ was applied before association testing.  This limited the number of \ac{SNP}s that qualified for testing to nine control-region \ac{SNP}s and seven coding-region \ac{SNP}s (See tables ~\ref{assoc.tab.cr} and ~\ref{assoc.tab.coding}, respectively).

One \ac{SNP} in the control region was selected for testing based on previous reports of its association with longevity in Italian\cite{zhang2003strikingly}, Finnish and Japanese\cite{niemi2005combination} populations.

\begin{figure}
\noindent\makebox[\textwidth]{%
  \includegraphics[width=0.85\textwidth]{fig/power_curve.png}}
  \caption[Statistical Power for Association Test]{
    \small{\textbf{Power to Detect Association.} Statistical power was calculated using PS software\cite{Dupont1990}. Curves are shown for the following control MAFs: 0.01 (blue), 0.05 (orange),0.10 (yellow), 0.25 (green) and 0.50 (purple).  For all curves, $\alpha = 0.05$ number of cases = 419, number of controls = 415.}}
  \label{assoc.fig.power}
\end{figure}

\section{Methods}\label{assoc.methods}

\subsection{Power Calculations}
Statistical power was calculated with the PS power and sample size calculator\cite{Dupont1990}.  Power curves were calculated for a sample of 419 cases and 415 controls, at minor allele frequencies of 0.01, 0.05, 0.10, 0.25 and 0.50, with a false-positive rate $\alpha = 0.05$.

\subsection{Genotyping and Quality Control}
Genotyping was performed on the Sequenom MassARRAY platform at the McGill University/Genome Qu\'{e}bec Innovation Centre.  A set of 92 SNPs were included in the first assay set.  A set of 37 genotyping assays were repeated due to quality control failure.

Quality control was performed in collaboration with Dr. Denise Daley (University of British Columbia, St.Paul's Hospital).  Assays with call rates below 95\% were considered `failed' and were re-designed.  Genotype cluster plots were visually inspected for irregularities.

\section{Results}
This study is powered to detect an odds ratio of at least 1.75 (or 0.45) for a variant at minor allele frequency of $\> 0.10$ with a false-positive rate of 0.05 (see ~\ref{assoc.fig.power}).

Of the 92 SNPs that were chosen for the initial round of Sequenom genotyping, 37 failed quality control (See ~\ref{appendix.table.repeated_snps}) due to low call rates.  These assays were re-designed and repeated.  Of the second set, only three assays failed quality controls (mt9947, rs41345446 and rs41347846).

After performing $\chi^{2}$ tests for association between mtDNA alleles and healthy aging, no variants that were tested showed association with the healthy aging phenotype, at a $p$-value significance threshold of 0.05.  The lowest $p$-value for control region SNPs was rs117135796 at position 152, with a $p$-value of 0.258 and odds ratio of 0.81.  For coding region SNPs, the lowest $p$-value was 0.280, with odds ratio 1.11 for rs2853495 at position 11,719 within the MT-ND4 gene.

The rs62581312 variant at position 150 within the control region showed a $p$-value of 0.171 and odds ratio of 0.72.

\begin{table}[!htb]
\begin{minipage}{\textwidth}
\caption{Mitochondrial Control Region (MAF $>$ 0.10)}
\label{assoc.tab.cr}
\noindent\makebox[\textwidth]{%
\footnotesize
\begin{tabular}{l r r c c c c c c c}
Chr &          ID & Position & Minor Allele & Major Allele & $F_{A}$\footnote{Minor Allele Frequency in `Affecteds' (seniors)} & $F_{U}$\footnote{Minor Allele Frequency in `Unaffecteds' (controls)} & $\chi^{2}$\footnote{$\chi^{2}$ test statistic} & $P$ & Odds Ratio \\ \hline 
  M &   rs3087742 &     73 &  A & G & 0.456 & 0.442 & 0.163 & 0.726 & 1.06 \\
  M & rs117135796 &    152 &  C & T & 0.185 & 0.218 & 1.405 & 0.258 & 0.81 \\
  M &   rs2857291 &    195 &  C & T & 0.170 & 0.181 & 0.173 & 0.714 & 0.93 \\
  M &  rs28625645 &    489 &  C & T & 0.102 & 0.098 & 0.031 & 0.908 & 1.04 \\
  M &     mt16126 & 16,126 &  C & T & 0.195 & 0.167 & 1.083 & 0.318 & 1.21 \\
  M &  rs55749223 & 16,189 &  C & T & 0.139 & 0.118 & 0.811 & 0.404 & 1.21 \\
  M &   rs2857290 & 16,270 &  T & C & 0.107 & 0.093 & 0.425 & 0.561 & 1.16 \\
  M &  rs34799580 & 16,311 &  C & T & 0.151 & 0.167 & 0.384 & 0.567 & 0.89 \\
  M &   rs3937033 & 16,519 &  T & C & 0.340 & 0.348 & 0.062 & 0.826 & 0.96 \\
\end{tabular}}
\end{minipage}
\end{table}

\begin{table}[!htb]
\begin{minipage}{\textwidth}
\caption{Mitochondrial Coding Region (MAF $>$ 0.10)}
\label{assoc.tab.coding}
\noindent\makebox[\textwidth]{%
\footnotesize
\begin{tabular}{l r r c c c c c c c}
Chr &         ID & Position & Minor Allele & Major Allele & $F_{A}$\footnote{Minor Allele Frequency in `Affecteds' (seniors)} & $F_{U}$\footnote{Minor Allele Frequency in `Unaffecteds' (controls)} & $\chi^{2}$\footnote{$\chi^{2}$ test statistic} & $P$ & Odds Ratio \\ \hline
  M &  rs2853517 &     709  & G & A & 0.144 & 0.128 & 0.942 & 0.332 & 1.14 \\
  M &  rs3928306 &   3,010  & C & T & 0.264 & 0.246 & 0.760 & 0.383 & 1.10 \\
  M &  rs2015062 &   7,028  & A & G & 0.445 & 0.436 & 0.155 & 0.694 & 1.04 \\
  M &  rs2853825 &   9,477  & G & A & 0.104 & 0.097 & 0.214 & 0.644 & 1.08 \\
  M &  rs2853495 &  11,719  & A & G & 0.493 & 0.468 & 1.167 & 0.280 & 1.11 \\
  M &  rs2853499 &  12,372  & C & T & 0.242 & 0.241 & 0.002 & 0.961 & 1.01 \\
  M & rs28357681 &  14,798  & A & G & 0.158 & 0.145 & 0.516 & 0.473 & 1.10 \\
\end{tabular}}
\end{minipage}
\end{table}

\begin{table}[!htb]
\begin{minipage}{\textwidth}
\caption{Replication of rs62581312 (C150T)}
\noindent\makebox[\textwidth]{%
\footnotesize
\begin{tabular}{l r r c c c c c c c}
Chr &          ID & Position & Minor Allele & Major Allele & $F_{A}$\footnote{Minor Allele Frequency in `Affecteds' (seniors)} & $F_{U}$\footnote{Minor Allele Frequency in `Unaffecteds' (controls)} & $\chi^{2}$\footnote{$\chi^{2}$ test statistic} & $P$ & Odds Ratio \\ \hline
  M &  rs62581312 &    150  & T & C & 0.082 & 0.110 & 1.88  & 0.171 & 0.72 \\
\end{tabular}}
\end{minipage}
\end{table}

\section{Discussion}
It is notable that this study did not replicate the previously-reported association at position 150 of the mitochondrial control region\cite{zhang2003strikingly}. There are several potential explanations for this result. The study by Zhang et al. focused on a group of Italians aged 99-106 years, whereas our samples qualify at age 85 and are mainly of British ancestry. Although the association was replicated in both Finnish and Japanese populations,\cite{niemi2005combination} there may be population-specific genetic or environmental factors that combine with the position 150 polymorphism to effect the aging phenotype.

Because the mitochondrial genome does not recombine, it is possible to identify sets of variants that are inherited together and form mitochondrial haplotypes.  These haplotypes have been traced to geographic/ancestral lineages across the world\cite{behar2007genographic}.  Previous studies have identified haplogroups that are associated with longevity\cite{dato2004association,Costa2009,de1999mitochondrial}.  In our study, we elected to combine a previously-published set of common European mitochondrial tag SNPs\cite{saxena2006comprehensive} with additional variants that were discovered by pooled next-generation sequencing.
% Force a new page
\newpage

\chapter{Discussion}\label{disc}

We have designed a cost-effective method of surveying the mitochondrial genomes of hundreds of samples for single-nucleotide polymorphisms.  Our method combines long-range PCR with a high-processivity, low-error DNA polymerase with pooled next-generation sequencing on the Illumina Genome Analyzer platform.  Our single-amplicon long-PCR mtDNA isolation method also eliminates complications due to co-amplification of of mtDNA-derived pseudogenes (NUMTs) in the nuclear genome.

While we have established that is is possible to isolate and sequence the whole mitochondrial genome via a single  long-PCR reaction, our mtDNA isolation protocol was not designed to detect common deletions that have been observed in other studies\cite{cortopassi1992pattern}.  Future studies may be able to take advantage of paired-end sequencing to detect relatively large-scale deletions such as the common 4.9 \ac{kb} deletion that has been characterized between \ac{rCRS} positions 8,470 and 13,446\cite{meissner20084977bp}.  In a paired-end sequencing experiment, deletions can be detected when paired reads map further apart than the expected $\sim 300$ bp insert size\cite{hajirasouliha2010detection}.

Previous reports have demonstrated accurate determination of allele frequencies of pooled genomic DNA on the ABI SOLiD, Roche 454 and Illumina GA IIx platforms\cite{druley2009quantification,Wei2011}.  Our estimation of MAF from Illumina sequencing of DNA pools correlates strongly with MAF calculated using genotypes determined using Sanger sequence data (Spearman's $r = 0.88$); this correlation is close to the value of $r^{2} = 0.9637$ published by Druley \textit{et al}\cite{druley2009quantification}.  The most likely source of discrepancy between these two datasets is due to small differences in the quantity of DNA that each sample contributes to the DNA pool.

Our study was not designed for sensitive detection of heteroplasmic variants, though we did observe a small number of variants that were suggestive of heteroplasmy via analysis of Sanger sequence traces.  These variants appear similar to heterozygous variants in diploid nuclear sequence data, with overlapping peaks of two different fluorophores (Fig ~\ref{var.fig.heteroplasmy}), and suggest a roughly equal mixture of two alleles.  For lower levels of heteroplasmy, it becomes difficult to distinguish true heteroplasmy from background noise in the Sanger sequence trace.  The goal of this study was to investigate the role that common, heritable mitochondrial variants may play in the human aging process.  Heteroplasmy can be inherited, and can also arise \textit{de novo}, and can vary by tissue type\cite{sondheimer2011neutral,coller2002frequent}.  Recent studies have shown that next-generation sequencing can be a powerful tool to detect heteroplasmy\cite{Li2010a}.  In order to detect heteroplasmic variants in a pooled sequencing experiment, one would need a way to tie each read to a specific sample, rather than estimate the allele frequencies of the whole pool as was done in our experiment.  New DNA barcoding methods (also called `indexed' sequencing) have now made this possible\cite{szelinger2011bar}.  It is well established that heteroplasmic variants accumulate with age\cite{sondheimer2011neutral,bender2006high,michikawa1999aging,calloway2000frequency}, so if we had used a sequencing technology that was sensitive to heteroplasmy then it is likely that we would have observed differences in levels in heteroplasmy between our cases ($> 85$ years of age) and controls (40\--54 years of age).  It would remain unclear, however, if those somatic heteroplasmic variants would be passed down to future generations and also to what extent heteroplasmic variants are involved with healthy aging.

In our analyses, MAF estimated from Illumina GA data is about 25\% lower than our measurement from Sanger sequencing.  We suggest that this discrepancy may represent a bias against mapping of reads containing non-reference bases.  We suggest that a read that contains a real non-reference base in the form of a SNP is less likely to align than a read that contains no non-reference SNPs, and that this problem will be increased in low-quality sequence data.  This phenomenon is referred to as `reference bias,' and has been observed in other next-generation sequencing experiments\cite{degner2009effect}.

When conducting a case-control genetic association study, it is important to control for possible population stratification.  If the case and control groups are composed of samples from different ethnic backgrounds, it is possible to observe false-positive associations due to differences in population-specific allele frequencies that play no functional role in the phenotype of interest.  Another study that is also part of the G$^{3}$ Study of Healthy Aging, and used the same sample set has analyzed a set of ancestry-informative markers and found no evidence for population stratification\cite{halaschek2012}.

Other studies have found evidence for gene-gene interactions in the etiology of type II diabetes mellitus.  One study used a non-parametric machine learning method known as Multifactor Dimensionality Reduction (MDR) to study genetic association with the metabolic disease.  Out of 23 loci on 15 candidate genes in the study, the researchers were able to identify a two-locus interaction between PPAR$\gamma$ and UCP2 that significantly reduced risk of T2DM in Koreans (odds ratio: 0.51, 95\% CI: {0.34, 0.77}, p=0.0016)\cite{cho2004multifactor}.  Another study, using a more traditional logistic regression model, identified a three-locus interaction between variants in UCP2, PGC-1$\alpha$ and position 10,398 of the mitochondrial genome in the North Indian Population\cite{bhat2007pgc}.  Although our study lacked the statistical power to detect these sorts of effects, this may be a fruitful direction for future studies of mitochondrial genetics in aging.
% Force a new page
\newpage

 
%    3. Notes
%    4. Footnotes
 
%    5. Bibliography
\begin{singlespace}
\raggedright
%\bibliographystyle{unsrt}
%\bibliographystyle{abbrvnat}
%\bibliographystyle{plainnat}
%\bibpunct{(}{)}{,}{n}{,}{;}
%\bibliography{biblio}
\printbibliography
 
\end{singlespace}
 
\appendix
%    6. Appendices (including copies of all required UBC Research
%       Ethics Board's Certificates of Approval)
\include{reb-coa}      % pdfpages is useful here
\chapter{Supplemental Figures}

\begin{landscape}
\begin{figure}
  \begin{center}
    \includegraphics[width=7in]{fig/coverage_graphs.pdf}
  \end{center}
  \caption[Short-Read Sequence Coverage of mtDNA]{
    \textbf{Sequence Coverage.} Sequence reads were aligned to the rCRS (\texttt{NC\_012920.1}) with MAQ.  Median coverage was 13,134 reads (31.3 reads per sample) for the case pool, and 12,683 reads (30.6 reads per sample) for the control pool.
  }
  \label{}
\end{figure}
\end{landscape}

\begin{landscape}
\begin{figure}
  \begin{center}
    \subfloat[Cases]   {\includegraphics[width=7in]{fig/sanger_maf_case.pdf}}\\
    \subfloat[Controls]{\includegraphics[width=7in]{fig/sanger_maf_cont.pdf}}
  \end{center}
  \caption[Control Region MAF Distribution from Sanger Sequencing]{
    \textbf{Minor Allele Frequencies from Sanger Dataset.}  A total of 277 SNPs were identified by Sanger sequencing.
  }
  \label{}
\end{figure}
\end{landscape}
\chapter{Mitochondrial Marker Data}
\input{notes/mitochondrial_genotyping/mito_marker_selection.tex}
\newpage
\input{notes/mitochondrial_genotyping/mito_set_01}
\newpage
\input{notes/mitochondrial_genotyping/mito_set_02}
\newpage
% \chapter{Nuclear Marker Data}
\input{notes/illumina_genotyping/tfam}
\newpage
\input{notes/illumina_genotyping/nrf1}
\newpage
\input{notes/illumina_genotyping/esrra}
\newpage
\input{notes/illumina_genotyping/gabpa}
\newpage
\input{notes/illumina_genotyping/mterf}
\newpage
\input{notes/illumina_genotyping/mterfd1}
\newpage
\input{notes/illumina_genotyping/polg}
\newpage
\input{notes/illumina_genotyping/polg2}
\newpage
\input{notes/illumina_genotyping/polrmt}
\newpage
\input{notes/illumina_genotyping/sp1}
\newpage
\input{notes/illumina_genotyping/tfb1m}
\newpage
\input{notes/illumina_genotyping/tfb2m}
\newpage
\input{notes/illumina_genotyping/pparg}
\newpage
 
\backmatter
%    7. Index
% See the makeindex package: the following page provides a quick overview
% <http://www.image.ufl.edu/help/latex/latex_indexes.shtml>

\end{document}
