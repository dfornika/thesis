\chapter{Discussion}\label{disc}

We have designed a cost-effective method of surveying the mitochondrial genomes of hundreds of samples for single-nucleotide polymorphisms.  Our method combines long-range PCR with a high-processivity, low-error DNA polymerase with pooled next-generation sequencing on the Illumina Genome Analyzer platform.  Our single-amplicon long-PCR mtDNA isolation method also eliminates complications due to co-amplification of of mtDNA-derived pseudogenes (NUMTs) in the nuclear genome.

While we have established that is is possible to isolate and sequence the whole mitochondrial genome via a single  long-PCR reaction, our mtDNA isolation protocol was not designed to detect common deletions that have been observed in other studies\cite{cortopassi1992pattern}.  Future studies may be able to take advantage of paired-end sequencing to detect relatively large-scale deletions such as the common 4.9 \ac{kb} deletion that has been characterized between \ac{rCRS} positions 8,470 and 13,446\cite{meissner20084977bp}.  In a paired-end sequencing experiment, deletions can be detected when paired reads map further apart than the expected $\sim 300$ bp insert size\cite{hajirasouliha2010detection}.

Previous reports have demonstrated accurate determination of allele frequencies of pooled genomic DNA on the ABI SOLiD, Roche 454 and Illumina GA IIx platforms\cite{druley2009quantification,Wei2011}.  Our estimation of MAF from Illumina sequencing of DNA pools correlates strongly with MAF calculated using genotypes determined using Sanger sequence data (Spearman's $r = 0.88$); this correlation is close to the value of $r^{2} = 0.9637$ published by Druley \textit{et al}\cite{druley2009quantification}.  The most likely source of discrepancy between these two datasets is due to small differences in the quantity of DNA that each sample contributes to the DNA pool.

Our study was not designed for sensitive detection of heteroplasmic variants, though we did observe a small number of variants that were suggestive of heteroplasmy via analysis of Sanger sequence traces.  These variants appear similar to heterozygous variants in diploid nuclear sequence data, with overlapping peaks of two different fluorophores (Fig ~\ref{var.fig.heteroplasmy}), and suggest a roughly equal mixture of two alleles.  For lower levels of heteroplasmy, it becomes difficult to distinguish true heteroplasmy from background noise in the Sanger sequence trace.  The goal of this study was to investigate the role that common, heritable mitochondrial variants may play in the human aging process.  Heteroplasmy can be inherited, and can also arise \textit{de novo}, and can vary by tissue type\cite{sondheimer2011neutral,coller2002frequent}.  Recent studies have shown that next-generation sequencing can be a powerful tool to detect heteroplasmy\cite{Li2010a}.  In order to detect heteroplasmic variants in a pooled sequencing experiment, one would need a way to tie each read to a specific sample, rather than estimate the allele frequencies of the whole pool as was done in our experiment.  New DNA barcoding methods (also called `indexed' sequencing) have now made this possible\cite{szelinger2011bar}.  It is well established that heteroplasmic variants accumulate with age\cite{sondheimer2011neutral,bender2006high,michikawa1999aging,calloway2000frequency}, so if we had used a sequencing technology that was sensitive to heteroplasmy then it is likely that we would have observed differences in levels in heteroplasmy between our cases ($> 85$ years of age) and controls (40\--54 years of age).  It would remain unclear, however, if those somatic heteroplasmic variants would be passed down to future generations and also to what extent heteroplasmic variants are involved with healthy aging.

In our analyses, MAF estimated from Illumina GA data is about 25\% lower than our measurement from Sanger sequencing.  We suggest that this discrepancy may represent a bias against mapping of reads containing non-reference bases.  We suggest that a read that contains a real non-reference base in the form of a SNP is less likely to align than a read that contains no non-reference SNPs, and that this problem will be increased in low-quality sequence data.  This phenomenon is referred to as `reference bias,' and has been observed in other next-generation sequencing experiments\cite{degner2009effect}.

When conducting a case-control genetic association study, it is important to control for possible population stratification.  If the case and control groups are composed of samples from different ethnic backgrounds, it is possible to observe false-positive associations due to differences in population-specific allele frequencies that play no functional role in the phenotype of interest.  Another study that is also part of the G$^{3}$ Study of Healthy Aging, and used the same sample set has analyzed a set of ancestry-informative markers and found no evidence for population stratification\cite{halaschek2012}.

Other studies have found evidence for gene-gene interactions in the etiology of type II diabetes mellitus.  One study used a non-parametric machine learning method known as Multifactor Dimensionality Reduction (MDR) to study genetic association with the metabolic disease.  Out of 23 loci on 15 candidate genes in the study, the researchers were able to identify a two-locus interaction between PPAR$\gamma$ and UCP2 that significantly reduced risk of T2DM in Koreans (odds ratio: 0.51, 95\% CI: {0.34, 0.77}, p=0.0016)\cite{cho2004multifactor}.  Another study, using a more traditional logistic regression model, identified a three-locus interaction between variants in UCP2, PGC-1$\alpha$ and position 10,398 of the mitochondrial genome in the North Indian Population\cite{bhat2007pgc}.  Although our study lacked the statistical power to detect these sorts of effects, this may be a fruitful direction for future studies of mitochondrial genetics in aging.
% Force a new page
\newpage
