%% The following is a directive for TeXShop to indicate the main file
%%!TEX root = diss.tex

\chapter{Abstract}

Mitochondria are thought to play a role in the aging process through their production of reactive oxygen species (ROS), and their regulation of cell fate via senescence and apoptosis.  We hypothesize that genetic variation in the mitochondrial genome may explain a portion of the phenotypic variance in the development of long-term good health.  To test this hypothesis, we have performed genetic association tests on a set of common mitochondrial polymorphisms, in a study of 419 exceptionally healthy seniors (cases) and 415 population-based mid-life individuals (controls).

Variant discovery was performed using Sanger sequencing of 834 individuals for the 1.1 kb non-coding mitochondrial control region, and identified 277 SNPs present in at least one individual.  A set of 92 mitochondrial coding-region SNPs were chosen via pooled high-throughput sequencing, combined with a previously-published set of European-specific mitochondrial tag SNPs.

After filtering for minor-allele frequency of $> 10\%$, a set of nine control-region SNPs and seven coding-region SNPs were tested for association with healthy aging.  None showed a statistically-significant association signal.  Additionally, one control-region variant that had shown association in an Italian centenarian population was tested in our sample set, but the association was not replicated.

