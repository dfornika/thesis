%% The following is a directive for TeXShop to indicate the main file
%%!TEX root = diss.tex

\chapter{Glossary}

% use \acrodef to define an acronym, but no listing
\acrodef{UI}{user interface}
\acrodef{UBC}{University of British Columbia}

% The acronym environment will typeset only those acronyms that were
% *actually used* in the course of the document
\begin{acronym}
\acro{ANOVA}[ANOVA]{Analysis of Variance\acroextra{, a set of
  statistical techniques to identify sources of variability between groups}}
\acro{ATP}{Adenosine Triphosphate}
\acro{BCGSC}{British Columbia Genome Sciences Centre}
\acro{CIHR}{the Canadian Institute for Health Research}
\acro{DNA}{Deoxyribonucleic Acid}
\acro{GA}{Genome Analyzer}
\acro{kb}{kilobase}
\acro{ETC}{Electron Transport Chain}
\acro{numt}{nuclear-mitochondrial sequences}
\acro{G3}{Genomics Genetics and Gerontology}
\acro{FMNH}{Flavin Mononucleotide}
\acro{FoGS}[FoGS]{The Faculty of Graduate Studies}
\acro{MAF}{Minor Allele Frequency}
\acro{MELAS}{Mitochondrial Encephalopathy Lactic Acidosis and Stroke-like episodes}
\acro{MERRF}{Myoclonic Epilepsy and Ragged-Red Fibres}
\acro{mtDNA}{Mitochondrial DNA}
\acro{NADH}{Nicotinamide Adenine Dinucleotide}
\acro{PCR}{Polymerase Chain Reaction}
\acro{LHON}{Leber's Hereditary Optic Neuropathy}
\acro{rCRS}{revised Cambridge Reference Sequence}
\acro{RNA}{Ribonucleic Acid}
\acro{ROS}{Reactive Oxygen Species}
\acro{RR}{Relative Risk}
\acro{rRNA}{Ribosomal RNA}
\acro{tRNA}{Transfer RNA}
\acro{OR}{Odds Ratio}
\acro{SNP}{Single Nucleotide Polymorphism}
\acro{URL}{Unique Resource Locator\acroextra{, used to describe a
    means for obtaining some resource on the world wide web}}
\end{acronym}

% You can also use \newacro{}{} to only define acronyms
% but without explictly creating a glossary
% 
% \newacro{ANOVA}[ANOVA]{Analysis of Variance\acroextra{, a set of
%   statistical techniques to identify sources of variability between groups.}}
% \newacro{API}[API]{application programming interface}
% \newacro{GOMS}[GOMS]{Goals, Operators, Methods, and Selection\acroextra{,
%   a framework for usability analysis.}}
% \newacro{TLX}[TLX]{Task Load Index\acroextra{, an instrument for gauging
%   the subjective mental workload experienced by a human in performing
%   a task.}}
% \newacro{UI}[UI]{user interface}
% \newacro{UML}[UML]{Unified Modelling Language}
% \newacro{W3C}[W3C]{World Wide Web Consortium}
% \newacro{XML}[XML]{Extensible Markup Language}
